\documentclass[12pt]{article}
\usepackage[utf8]{inputenc}
\usepackage[brazil]{babel}
\usepackage{graphicx}
\usepackage{url}

\begin{document}

\begin{titlepage}
\centering

{\Huge \textbf{Centro Universitário Católica de Santa Catarina}}\\[1.5cm]
{\Large Curso de Engenharia de Software}\\[4cm]
{\huge \textbf{FLEETTRACK}}\\[2cm]
{\Large \textbf{Gabriel Knopka Ramos}}\\[6cm]
{\Large Joinville, SC}\\[1.5cm]
{\Large 2025}

\vfill

\end{titlepage}

\newpage

\begin{abstract}
\noindent \textbf{Resumo:} Este trabalho apresenta o desenvolvimento de um Sistema de Controle de Frotas, denominado FleetTrack, voltado para tornar a gestão de veículos mais simples e eficiente. O projeto busca oferecer uma solução prática e acessível para empresas que desejam organizar melhor suas operações, reduzir custos e ter mais controle sobre suas atividades diárias.
\end{abstract}

\vspace{1cm}

\section*{1 Introdução}
Gerenciar frotas de veículos é um desafio constante para empresas de transporte, logística e outros setores que dependem desse tipo de operação. À medida que as demandas crescem, a necessidade de ferramentas modernas que facilitem essa tarefa também aumenta. Pensando nisso, este projeto propõe o desenvolvimento de um sistema focado em otimizar o dia a dia das empresas, promovendo mais organização, controle e eficiência.

O objetivo é criar uma ferramenta que ajude na gestão de veículos de forma prática, oferecendo recursos como o cadastro de veículos e motoristas, geração de relatórios de uso e desempenho, além do envio de alertas para situações que exigem atenção. Com isso, espera-se que o FleetTrack se torne um aliado importante no planejamento e na execução de atividades relacionadas à frota, impactando positivamente a logística e os resultados estratégicos das empresas.

\newpage

\section*{2 Descrição do Projeto}
O FleetTrack é pensado para empresas que precisam de uma gestão mais ágil e centralizada de suas frotas. A plataforma permitirá o registro de informações importantes sobre veículos e motoristas, a geração de relatórios de desempenho e o envio de alertas para situações críticas, tudo de maneira simples e acessível.

Muitas empresas enfrentam dificuldades como falta de dados consolidados, dificuldade de organizar as rotas, altos custos com operações ineficientes e consumo excessivo de combustível. Além disso, a ausência de um sistema dedicado pode comprometer a segurança e o planejamento. A proposta do FleetTrack é justamente enfrentar esses desafios, oferecendo uma solução que melhore o acompanhamento das operações, ajude a otimizar trajetos e promova um uso mais inteligente dos recursos.

Vale ressaltar que o projeto possui limitações: ele não incluirá funcionalidades de manutenção preditiva dos veículos nem integração com sensores mecânicos. O foco é atender exclusivamente a frotas terrestres, sem considerar operações aéreas ou marítimas. Assim, concentramos os esforços na eficiência operacional e na melhoria do suporte estratégico às empresas.

\newpage

\section*{3 Especificação Técnica}
Este projeto propõe a criação de um sistema de gestão de frotas que prioriza a organização e a eficiência, voltado para empresas que precisam de maior controle sobre seus veículos e operações.

Entre as principais funcionalidades estão: cadastro de veículos e motoristas, visualização de dados operacionais, geração de relatórios detalhados sobre a frota e alertas para eventos importantes. O objetivo é oferecer uma plataforma que seja prática, segura e que ajude as empresas a tomarem decisões mais assertivas.

\subsection*{3.1 Requisitos de Software}

\subsubsection*{Requisitos Funcionais (RF)}
\begin{itemize}
\item RF01: Permitir o cadastro e gerenciamento de veículos da frota.
\item RF02: Permitir o registro de motoristas e suas informações.
\item RF03: Gerar relatórios de desempenho da frota.
\item RF04: Permitir a criação e o gerenciamento de usuários com diferentes níveis de acesso.
\item RF05: Fornecer um painel administrativo para acompanhamento geral da operação.
\end{itemize}

\subsubsection*{Requisitos Não Funcionais (RNF)}
\begin{itemize}
\item RNF01: O sistema deve ser acessível via web.
\item RNF02: A plataforma deve garantir respostas rápidas às solicitações dos usuários.
\item RNF03: Os dados armazenados devem ser mantidos por no mínimo seis meses.
\item RNF04: A interface do sistema deve ser intuitiva e amigável.
\item RNF05: O sistema deve gerar notificações em casos de eventos críticos, como atrasos ou desvios de padrão.
\item RNF06: O sistema deve possibilitar a exportação de relatórios de desempenho operacional.
\end{itemize}

\subsection*{3.2 Considerações de Design}
A arquitetura do sistema será pensada para garantir escalabilidade, alta disponibilidade e facilidade de manutenção. Será considerada a abordagem de microsserviços para modularizar as funcionalidades, garantindo uma estrutura robusta e flexível.

O sistema será composto por três camadas principais:
\begin{itemize}
\item \textbf{Frontend:} Responsável pela interação com o usuário via navegador.
\item \textbf{Backend:} Gerenciará a lógica de negócios e o processamento de dados.
\item \textbf{Banco de Dados:} Armazenará informações sobre veículos, motoristas, rotas e relatórios.
\end{itemize}

\subsubsection*{Padrões de Arquitetura}
Será utilizado o padrão MVC (Model-View-Controller) para organizar o backend, favorecendo a separação de responsabilidades. A comunicação entre as camadas será feita por meio de APIs RESTful.

\subsubsection*{Modelos C4}
A estrutura do sistema será descrita utilizando o Modelo C4, abrangendo:
\begin{itemize}
\item \textbf{Contexto:} Visão geral da interação entre o sistema, usuários e serviços externos.
\item \textbf{Contêineres:} Organização dos principais componentes do sistema.
\item \textbf{Componentes:} Detalhamento dos serviços e módulos internos.
\item \textbf{Código:} Estrutura de implementação das principais funcionalidades.
\end{itemize}

\subsection*{3.3 Stack Tecnológica}

\subsubsection*{Linguagens de Programação:}
\begin{itemize}
\item \textbf{Frontend:} Django Templates (HTML, CSS, JavaScript) para renderização dinâmica no lado do servidor.
\item \textbf{Backend:} Django (Python) para desenvolvimento da aplicação.
\item \textbf{Banco de Dados:} MySQL para armazenamento estruturado de informações.
\end{itemize}

\subsubsection*{Ferramentas de Desenvolvimento e Gestão de Projeto:}
\begin{itemize}
\item Git/GitHub para controle de versão e colaboração.
\item Postman para testes de APIs.
\item Docker para containerização e ambiente replicável.
\end{itemize}

\subsection*{3.4 Considerações de Segurança}

A segurança será um ponto central no desenvolvimento do sistema. Algumas medidas previstas:
\begin{itemize}
\item \textbf{Autenticação e autorização:} Controle de acesso utilizando Django Authentication com JWT para APIs.
\item \textbf{Proteção contra vulnerabilidades:} Medidas para evitar ataques como SQL Injection, Cross-Site Scripting (XSS) e Cross-Site Request Forgery (CSRF).
\end{itemize}

\newpage

\section*{4 PRÓXIMOS PASSOS}
Para garantir um desenvolvimento organizado e validado em todas as fases do projeto, os próximos passos estão estruturados conforme descrito a seguir:

\subsection{Criação de Mockups e Fluxos de Navegação}
\begin{itemize}
    \item Elaboração dos mockups das telas principais do sistema.
    \item Definição dos fluxos de navegação entre as páginas.
\end{itemize}

\subsection{Desenvolvimento de Protótipo}
\begin{itemize}
    \item Implementação de uma versão inicial com funcionalidades básicas.
    \item Testes preliminares com foco na navegação e fluxo de dados.
    \item Correção de eventuais falhas de usabilidade ou lógica.
\end{itemize}

\subsection{Implementação dos Módulos}
\begin{itemize}
    \item Desenvolvimento dos principais componentes do sistema:
    \begin{itemize}
        \item Cadastro de motoristas e veículos;
        \item Gerenciamento de viagens e notificações;
        \item Emissão de relatórios operacionais.
    \end{itemize}
    \item Aplicação de boas práticas de segurança e desempenho.
\end{itemize}

\subsection{Testes e Validações}
\begin{itemize}
    \item Realização de testes unitários e de integração.
    \item Validação de funcionalidades conforme requisitos do projeto.
    \item Ajustes com base em feedback dos testes.
\end{itemize}

\newpage

\section*{Referências}
\begin{itemize}
\item \textbf{Django.} The web framework for perfectionists with deadlines. Disponível em: \url{https://www.djangoproject.com}. Acesso em: 14 mai. 2025.
\item \textbf{MySQL.} MySQL: The world's most advanced open source relational database. Disponível em: \url{https://www.mysql.com}. Acesso em: 8 abr. 2025.
\item \textbf{JWT.io.} JWT (JSON Web Token) Introduction. Disponível em: \url{https://jwt.io}. Acesso em: 19 mai. 2025.
\item \textbf{Docker.} Docker - Empowering developers to write, test and deploy code anywhere. Disponível em: \url{https://www.docker.com}. Acesso em: 3 jun. 2025.
\item \textbf{GitHub.} GitHub - Git repository hosting service. Disponível em: \url{https://github.com}. Acesso em: 29 jul. 2025.
\item \textbf{Postman.} Postman - API platform for building and using APIs. Disponível em: \url{https://www.postman.com}. Acesso em: 20 nov. 2025.
\end{itemize}

\end{document}
