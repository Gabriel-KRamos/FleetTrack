\documentclass[12pt]{article}
\usepackage{float}
\usepackage[utf8]{inputenc}
\usepackage[brazil]{babel}
\usepackage{graphicx}
\usepackage{url}
\renewcommand{\labelitemiii}{•}      

\begin{document}

\begin{titlepage}
\centering

{\Huge \textbf{Centro Universitário Católica de Santa Catarina}}\\[1.5cm]
{\Large Curso de Engenharia de Software}\\[4cm]
{\huge \textbf{FLEETTRACK}}\\[2cm]
{\Large \textbf{Gabriel Knopka Ramos}}\\[6cm]
{\Large Joinville, SC}\\[1.5cm]
{\Large 2025}

\vfill

\end{titlepage}

\newpage

\begin{abstract}
\noindent \textbf{Resumo:} Este trabalho apresenta o desenvolvimento de um Sistema de Controle de Frotas, denominado FleetTrack, voltado para tornar a gestão de veículos mais simples e eficiente. O projeto busca oferecer uma solução prática e acessível para empresas que desejam organizar melhor suas operações, reduzir custos e ter mais controle sobre suas atividades diárias.
\end{abstract}

\vspace{1cm}

\section{introdução}
\noindent \textbf{Contexto:} A gestão de frotas é uma atividade fundamental para empresas que dependem do transporte de mercadorias ou prestação de serviços. O controle eficiente dos veículos e motoristas, bem como o acompanhamento das manutenções, são fatores essenciais para garantir segurança, reduzir custos operacionais e otimizar a logística.

\noindent \textbf{Justificativa:} A criação de uma ferramenta especializada para a gestão de veículos se mostra altamente relevante no campo da engenharia de software, pois oferece uma solução tecnológica capaz de automatizar processos, reduzir falhas humanas e melhorar a tomada de decisão. O sistema proposto busca atender a essa necessidade, alinhando-se às boas práticas de desenvolvimento e ao uso estratégico de tecnologias para resolver problemas reais.

\noindent \textbf{Objetivos:} O objetivo principal do projeto é criar uma ferramenta que ajude na gestão de veículos de forma prática, oferecendo recursos como o cadastro de veículos e motoristas, geração de relatórios de uso e desempenho, além do envio de alertas para situações que exigem atenção. Com isso, espera-se que o FleetTrack se torne um aliado importante no planejamento e na execução de atividades relacionadas à frota, impactando positivamente a logística e os resultados estratégicos das empresas.

\newpage

\section{descrição do projeto}

O FleetTrack é pensado para empresas que precisam de uma gestão mais ágil e centralizada de suas frotas. A plataforma permitirá o registro de informações importantes sobre veículos e motoristas, a geração de relatórios de desempenho e o envio de alertas para situações críticas, tudo de maneira simples e acessível.

Muitas empresas enfrentam dificuldades como falta de dados consolidados, dificuldade de organizar as rotas, altos custos com operações ineficientes e consumo excessivo de combustível. Além disso, a ausência de um sistema dedicado pode comprometer a segurança e o planejamento. A proposta do FleetTrack é justamente enfrentar esses desafios, oferecendo uma solução que melhore o acompanhamento das operações, ajude a otimizar trajetos e promova um uso mais inteligente dos recursos.

Vale ressaltar que o projeto possui limitações: ele não incluirá funcionalidades de manutenção preditiva dos veículos nem integração com sensores mecânicos. O foco é atender exclusivamente a frotas terrestres, sem considerar operações aéreas ou marítimas. Assim, concentramos os esforços na eficiência operacional e na melhoria do suporte estratégico às empresas.

\newpage

\section{especificação técnica}

Este projeto propõe a criação de um sistema de gestão de frotas que prioriza a organização e a eficiência, voltado para empresas que precisam de maior controle sobre seus veículos e operações.

Entre as principais funcionalidades estão: cadastro de veículos e motoristas, visualização de dados operacionais, geração de relatórios detalhados sobre a frota e alertas para eventos importantes. O objetivo é oferecer uma plataforma que seja prática, segura e que ajude as empresas a tomarem decisões mais assertivas.


\subsection{requisitos de software}
   \textbf{3.1.1 Requisitos funcionais (rf)}
\begin{itemize}
    \item \textbf{RF1 - Cadastro e Organização de Veículos}
    \begin{itemize}
        \item RF1.1: Permitir o cadastro de veículos, com informações como: placa, modelo, ano de fabricação, capacidade e status operacional.
        \item RF1.2: Permitir o cadastro de informações adicionais sobre o estado de manutenção do veículo:
        \begin{itemize}
            \item Saúde dos pneus.
            \item Data e quilometragem da última troca de óleo.
            \item Quilometragem atual.
            \item Data e descrição da última revisão mecânica.
            \item Data prevista para próxima manutenção preventiva.
        \end{itemize}
        \item RF1.3: Permitir o cadastro de destinos.
        \item RF1.4: Atribuir um destino ou rota específica a um veículo.
        \item RF1.5: Permitir a atualização de informações de veículos cadastrados.
        \item RF1.6: Permitir a exclusão ou desativação de veículos fora de operação.
    \end{itemize}

    \item \textbf{RF2 - Cadastro e Organização de Motoristas}

    \begin{itemize}
        \item RF02.1: Permitir o cadastro de motoristas com informações pessoais, documentação e histórico de viagens.
        \item RF2.2: Permitir a atualização de informações dos motoristas.
        \item RF2.3: Permitir a exclusão ou desativação de motoristas inativos.
    \end{itemize}

    \item \textbf{RF3 - Monitoramento e Relatórios Operacionais}

    \begin{itemize}
        \item RF3.1: Permitir a geração de relatórios de desempenho por veículo e motorista.
        \item RF3.2: Exibir relatórios com métricas como:
        \begin{itemize}
            \item Quilometragem percorrida.
            \item Número de viagens realizadas.
            \item Consumo médio de combustível.
            \item Estado atual dos componentes do veículo.
            \item Proximidade das próximas manutenções.
        \end{itemize}
        \item RF3.3: Ao atribuir uma viagem a um veículo, exibir uma previsão de desgaste, considerando:
        \begin{itemize}
            \item Quilometragem estimada da viagem.
            \item Impacto na saúde dos pneus.
            \item Aproximação de revisões, trocas de óleo e outros itens críticos.
        \end{itemize}
        \item RF3.4: Permitir exportação dos relatórios.
    \end{itemize}

    \item \textbf{RF4 - Gestão de Usuários e Controle de Acesso}

    \begin{itemize}
        \item RF4.1: Implementar sistema de autenticação com credenciais individuais.
        \item RF4.2: Criar diferentes níveis de acesso.
    \end{itemize}

    \item \textbf{RF5 - Painel Administrativo e Alertas}

    \begin{itemize}
        \item RF5.1: Disponibilizar um painel com visão geral das operações:
        \begin{itemize}
            \item Quantidade de veículos ativos.
            \item Motoristas disponíveis.
            \item Viagens em andamento.
            \item Status de manutenção dos veículos.
        \end{itemize}
        \item RF5.2: Implementar sistema de alertas automáticos para eventos críticos:
        \begin{itemize}
            \item Aproximação da data de revisão, troca de óleo ou necessidade de troca de pneus.
        \end{itemize}
    \end{itemize}
\end{itemize}
\begin{figure}[!hbtp]
    \centering
    \includegraphics[width=1.2\linewidth]{UML1FleetTrack.png}
    \caption{3.0 UML Administrador}
    \label{fig:uml-admin}
\end{figure}
\begin{figure}[!htp]
    \centering
    \includegraphics[width=0.8\linewidth]{UML2FleetTrack.png}
    \caption{3.0 UML Motorista}
    \label{fig:uml-motorista}
\end{figure}

\newpage
\textbf{3.1.2 Requisitos Não Funcionais}
\begin{itemize}
\item RNF01: O sistema deve ser acessível via web.
\item RNF02: A plataforma deve garantir respostas rápidas às solicitações dos usuários.
\item RNF03: Os dados armazenados devem ser mantidos por no mínimo seis meses.
\item RNF04: A interface do sistema deve ser intuitiva e amigável.
\item RNF05: O sistema deve gerar notificações em casos de eventos críticos, como atrasos ou desvios de padrão.
\end{itemize}

\subsection{Considerações de design}

\subsubsection*{3.2.1 Discussão sobre as escolhas de design}

Durante o planejamento do FleetTrack, foram consideradas diversas alternativas de design para garantir uma solução escalável, segura e de fácil manutenção. A adoção de uma arquitetura baseada em três camadas (frontend, backend e banco de dados) foi preferida devido à sua simplicidade e eficiência em projetos. 

Inicialmente, cogitou-se o uso de uma abordagem monolítica, pela facilidade de desenvolvimento e implantação, especialmente em projetos acadêmicos. Contudo, optou-se pela modularização através de microsserviços, visando maior escalabilidade e isolamento de falhas, facilitando a manutenção e futuras expansões.

O padrão MVC (Model-View-Controller) foi escolhido para o backend devido à sua ampla adoção, clareza na separação de responsabilidades e compatibilidade nativa com o framework Django.

\subsubsection*{3.2.2 Visão Inicial da Arquitetura}

A arquitetura do FleetTrack será composta pelos seguintes componentes principais e suas respectivas interconexões:

\begin{itemize}
    \item \textbf{Frontend:} Interface web desenvolvida com Django Templates, HTML, CSS e JavaScript, responsável pela interação com os usuários. Comunica-se com o backend via chamadas HTTP.
    \item \textbf{Backend:} API e lógica de negócios desenvolvida em Django, responsável pelo processamento de dados, regras de negócio e geração de relatórios.
    \item \textbf{Banco de Dados:} MySQL, destinado ao armazenamento persistente de informações relativas aos veículos, motoristas, trajetos e relatórios.
    \item \textbf{Serviço de Autenticação:} Implementação de autenticação e autorização baseada em JWT (JSON Web Token), garantindo segurança na troca de informações.
\end{itemize}

A comunicação entre frontend e backend será realizada via APIs RESTful, utilizando o padrão HTTP, enquanto o backend se conecta ao banco de dados através do ORM (Object-Relational Mapping) do Django.

\subsubsection*{3.2.3 Padrões de Arquitetura}

O projeto adota os seguintes padrões arquiteturais:

\begin{itemize}
    \item \textbf{MVC (Model-View-Controller):} Para organizar a estrutura do backend, separando claramente as camadas de modelo de dados, lógica de negócio e apresentação.
    \item \textbf{RESTful APIs:} Para padronizar a comunicação entre frontend e backend, facilitando a integração com outros sistemas.
\end{itemize}

\subsubsection*{3.2.4 Modelos C4}

Para uma melhor compreensão da arquitetura do FleetTrack, utilizamos o Modelo C4, que descreve o sistema em quatro níveis de abstração:

\paragraph{Contexto}

O FleetTrack atua como um sistema de gestão de frotas acessado via web por diferentes perfis de usuários: administradores, operadores e motoristas.

\begin{itemize}
    \item \textbf{Usuários:} Administradores (gestão e relatórios), motoristas (visualização de itinerários).
\end{itemize}
\begin{figure}[!ht]
    \centering
    \includegraphics[width=0.7\linewidth]{C4Contexto.png}
    \caption{3.2 C4 Contexto}
    \label{fig::C4 Contexto}
\end{figure}
 \newpage
 
\paragraph{Contêineres}
O sistema é dividido em três contêineres principais:

\begin{itemize}
    \item \textbf{Frontend Web:} Aplicação baseada em Django Templates com HTML/CSS/JS. Responsável por renderizar páginas e interagir com o backend via requisições HTTP.
    \item \textbf{Backend API:} Serviço em Django, que expõe endpoints RESTful, processa as regras de negócio, autentica usuários via JWT e interage com o banco de dados.
    \item \textbf{Banco de Dados Relacional:} MySQL, responsável pelo armazenamento persistente e seguro dos dados da frota.
\end{itemize}

\paragraph{Componentes}

Cada contêiner possui componentes específicos, conforme descrito a seguir:

\begin{itemize}
    \item \textbf{Frontend:}
    \begin{itemize}
        \item Sistema de autenticação e login.
        \item Painel administrativo com visualização de indicadores.
        \item Telas de cadastro e edição de motoristas e veículos.
    \end{itemize}
    \newpage
    \item \textbf{Backend:}
    \begin{itemize}
        \item Serviço de autenticação e geração de tokens JWT.
        \item Módulo de cadastro de motoristas e veículos.
        \item Módulo de geração de relatórios.
        \item Sistema de alertas e notificações.
    \end{itemize}
    
    \item \textbf{Banco de Dados:}
    \begin{itemize}
        \item Tabela \texttt{usuarios}: armazena dados de autenticação e perfis.
        \item Tabela \texttt{veiculos}: informações detalhadas sobre os veículos.
        \item Tabela \texttt{motoristas}: dados cadastrais e de habilitação.
        \item Tabela \texttt{rotas}: armazena dados de rotas utilizadas.
    \end{itemize}
\end{itemize}
\begin{figure}[!ht]
    \centering
    \includegraphics[width=0.9\linewidth]{C4Componentes.png}
    \caption{3.2 C4 Componentes}
    \label{fig::C4 Componentes}
\end{figure}
 \newpage
\subsection{Stack Tecnológica}

\subsubsection*{Linguagens de Programação:}
\begin{itemize}
\item \textbf{Frontend:} Django Templates (HTML, CSS, JavaScript) para renderização dinâmica no lado do servidor.
\item \textbf{Backend:} Django (Python) para desenvolvimento da aplicação.
\item \textbf{Banco de Dados:} MySQL para armazenamento estruturado de informações.
\end{itemize}

\subsubsection*{Ferramentas de Desenvolvimento e Gestão de Projeto:}
\begin{itemize}
\item Git/GitHub para controle de versão e colaboração.
\item Postman para testes de APIs.
\item Docker para containerização e ambiente replicável.
\end{itemize}

\subsection{Considerações de Segurança}

A segurança será um ponto central no desenvolvimento do sistema. Algumas medidas previstas:
\begin{itemize}
\item \textbf{Autenticação e autorização:} Controle de acesso utilizando Django Authentication com JWT para APIs.
\item \textbf{Proteção contra vulnerabilidades:} Medidas para evitar ataques como SQL Injection, Cross-Site Scripting (XSS) e Cross-Site Request Forgery (CSRF).
\end{itemize}

\newpage

\section{Próximos Passos}

Para garantir um desenvolvimento organizado e validado em todas as fases do projeto, os próximos passos estão estruturados conforme descrito a seguir:

\subsection{Criação de mockups e fluxos de navegação}
\begin{itemize}
    \item Elaboração dos mockups das telas principais do sistema.
    \item Definição dos fluxos de navegação entre as páginas.
\end{itemize}

\subsection{Desenvolvimento de protótipo}
\begin{itemize}
    \item Implementação de uma versão inicial com funcionalidades básicas.
    \item Testes preliminares com foco na navegação e fluxo de dados.
    \item Correção de eventuais falhas de usabilidade ou lógica.
\end{itemize}

\subsection{Implementação dos módulos}
\begin{itemize}
    \item Desenvolvimento dos principais componentes do sistema:
    \begin{itemize}
        \item Cadastro de motoristas e veículos;
        \item Gerenciamento de viagens e notificações;
        \item Emissão de relatórios operacionais.
    \end{itemize}
    \item Aplicação de boas práticas de segurança e desempenho.
\end{itemize}

\subsection{Testes e validações}
\begin{itemize}
    \item Realização de testes unitários e de integração.
    \item Validação de funcionalidades conforme requisitos do projeto.
    \item Ajustes com base em feedback dos testes.
\end{itemize}

\newpage

\section{Referências}
\begin{itemize}
\item \textbf{Django.} The web framework for perfectionists with deadlines. Disponível em: \url{https://www.djangoproject.com}. Acesso em: 14 mai. 2025.
\item \textbf{MySQL.} MySQL: The world's most advanced open source relational database. Disponível em: \url{https://www.mysql.com}. Acesso em: 8 abr. 2025.
\item \textbf{JWT.io.} JWT (JSON Web Token) Introduction. Disponível em: \url{https://jwt.io}. Acesso em: 19 mai. 2025.
\item \textbf{Docker.} Docker - Empowering developers to write, test and deploy code anywhere. Disponível em: \url{https://www.docker.com}. Acesso em: 3 jun. 2025.
\item \textbf{GitHub.} GitHub - Git repository hosting service. Disponível em: \url{https://github.com}. Acesso em: 29 jul. 2025.
\item \textbf{Postman.} Postman - API platform for building and using APIs. Disponível em: \url{https://www.postman.com}. Acesso em: 20 nov. 2025.
\end{itemize}

\end{document}
