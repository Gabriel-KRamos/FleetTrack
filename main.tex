\documentclass[12pt]{article}
\usepackage{float}
\usepackage[utf8]{inputenc}
\usepackage[brazil]{babel}
\usepackage{graphicx}
\usepackage{url}
\renewcommand{\labelitemiii}{•}      

\begin{document}

\begin{titlepage}
\centering

{\Huge \textbf{Centro Universitário Católica de Santa Catarina}}\\[1.5cm]
{\Large Curso de Engenharia de Software}\\[4cm]
{\huge \textbf{FLEETTRACK}}\\[2cm]
{\Large \textbf{Gabriel Knopka Ramos}}\\[6cm]
{\Large Joinville, SC}\\[1.5cm]
{\Large 2025}

\vfill

\end{titlepage}

\newpage

\begin{abstract}
\noindent \textbf{Resumo:} Este trabalho apresenta o desenvolvimento de um Sistema de Controle de Frotas, denominado FleetTrack, voltado para tornar a gestão de veículos mais simples e eficiente. O projeto busca oferecer uma solução prática e acessível para empresas que desejam organizar melhor suas operações, reduzir custos e ter mais controle sobre suas atividades diárias.
\end{abstract}

\vspace{1cm}

\section{introdução}
\noindent \textbf{Contexto:} A gestão de frotas é uma atividade fundamental para empresas que dependem do transporte de mercadorias ou prestação de serviços. O controle eficiente dos veículos e motoristas, bem como o acompanhamento das manutenções, são fatores essenciais para garantir segurança, reduzir custos operacionais e otimizar a logística.

\noindent \textbf{Justificativa:} A criação de uma ferramenta especializada para a gestão de veículos se mostra altamente relevante no campo da engenharia de software, pois oferece uma solução tecnológica capaz de automatizar processos, reduzir falhas humanas e melhorar a tomada de decisão. O sistema proposto busca atender a essa necessidade, alinhando-se às boas práticas de desenvolvimento e ao uso estratégico de tecnologias para resolver problemas reais.

\noindent \textbf{Objetivos:} O objetivo principal do projeto é criar uma ferramenta que ajude na gestão de veículos de forma prática, oferecendo recursos como o cadastro de veículos e motoristas, geração de relatórios de uso e desempenho, além do envio de alertas para situações que exigem atenção. Com isso, espera-se que o FleetTrack se torne um aliado importante no planejamento e na execução de atividades relacionadas à frota, impactando positivamente a logística e os resultados estratégicos das empresas.

\newpage

\section{descrição do projeto}

O FleetTrack é pensado para empresas que precisam de uma gestão mais ágil e centralizada de suas frotas. A plataforma permitirá o registro de informações importantes sobre veículos e motoristas, a geração de relatórios de desempenho e o envio de alertas para situações críticas, tudo de maneira simples e acessível.

Muitas empresas enfrentam dificuldades como falta de dados consolidados, dificuldade de organizar as rotas, altos custos com operações ineficientes e consumo excessivo de combustível. Além disso, a ausência de um sistema dedicado pode comprometer a segurança e o planejamento. A proposta do FleetTrack é justamente enfrentar esses desafios, oferecendo uma solução que melhore o acompanhamento das operações, ajude a otimizar trajetos e promova um uso mais inteligente dos recursos.

Vale ressaltar que o projeto possui limitações: ele não incluirá funcionalidades de manutenção preditiva dos veículos nem integração com sensores mecânicos. O foco é atender exclusivamente a frotas terrestres, sem considerar operações aéreas ou marítimas. Assim, concentramos os esforços na eficiência operacional e na melhoria do suporte estratégico às empresas.

\newpage

\section{especificação técnica}

Este projeto propõe a criação de um sistema de gestão de frotas que prioriza a organização e a eficiência, voltado para empresas que precisam de maior controle sobre seus veículos e operações.

Entre as principais funcionalidades estão: cadastro de veículos e motoristas, visualização de dados operacionais, geração de relatórios detalhados sobre a frota e alertas para eventos importantes. O objetivo é oferecer uma plataforma que seja prática, segura e que ajude as empresas a tomarem decisões mais assertivas.


\subsection{requisitos de software}
   \textbf{3.1.1 Requisitos funcionais}
\begin{itemize}
    \item \textbf{RF1 - Cadastro e Organização de Veículos}
    \begin{itemize}
        \item RF1.1: O sistema deve permitir o cadastro de veículos, com informações
        \item RF1.2: O sistema deve permitir o cadastro de informações adicionais sobre o estado de manutenção do veículo
        \item RF1.3: O sistema deve permitir o cadastro de rotas.
        \item RF1.4: Atribuir um destino ou rota específica a um veículo.
        \item RF1.5: Permitir a atualização de informações de veículos cadastrados.
        \item RF1.6: Permitir a exclusão ou desativação de veículos fora de operação.
    \end{itemize}

    \item \textbf{RF2 - Cadastro e Organização de Motoristas}

    \begin{itemize}
        \item RF02.1: O sistema deve permitir o cadastro de motoristas
        \begin{itemize}
        \end{itemize}
        \item RF2.2: Permitir a atualização de informações dos motoristas.
    \end{itemize}

    \item \textbf{RF3 - Monitoramento e Relatórios Operacionais}

    \begin{itemize}
        \item RF3.1: Ao atribuir uma viagem a um veículo, exibir uma previsão de gastos
        \begin{itemize}
        \end{itemize}
    \end{itemize}

    \item \textbf{RF4 - Painel Administrativo e Alertas}

    \begin{itemize}
        \item RF4.1: Disponibilizar um painel com visão geral das operações
        \begin{itemize}
        \end{itemize}
        \item RF4.2: Implementar sistema de alertas automáticos para eventos crítico
        \end{itemize}
    \end{itemize}
\end{itemize}
O Diagrama de Casos de Uso(Figura 1) do sistema representa, de forma estruturada, as funcionalidades que podem ser acessadas pelo perfil Administrador no sistema, categorizadas por domínio funcional.

A seção Gestão de Usuários contempla ações administrativas relacionadas ao ciclo de vida de usuários do sistema, como criação, desativação, atualização de dados e gerenciamento de níveis de acesso. Na Gestão de Motoristas, o administrador pode realizar o cadastro, atualização e desativação de motoristas da frota.

A área Painel e Alertas fornece acesso ao painel administrativo e a alertas críticos, os quais são fundamentais para a tomada de decisão em tempo real. Por fim, a Gestão de Veículos permite operações completas sobre a frota, incluindo cadastro e atualização de veículos, desativação, atribuição de rotas, controle de manutenção, consulta de status e definição de destinos.

Este diagrama evidencia o papel central do administrador no sistema, tendo acesso completo às funcionalidades críticas que garantem o controle operacional e gerencial da frota.

\begin{figure}[!hbtp]
    \centering
    \includegraphics[width=0.8\linewidth,height=0.8\textheight,keepaspectratio]{casodeUso.png}
    \caption{UML Administrador}
    \label{fig:uml-admin}
\end{figure}

\newpage
\textbf{3.1.2 Requisitos Não Funcionais}
\begin{itemize}
\item RNF01: O sistema deve ser acessível via web.
\item RNF02: A plataforma deve garantir respostas rápidas às solicitações dos usuários.
\item RNF03: Os dados armazenados devem ser mantidos por no mínimo seis meses.
\item RNF04: A interface do sistema deve ser intuitiva e amigável.
\end{itemize}

\subsection{Considerações de design}

\subsubsection*{3.2.1 Discussão sobre as escolhas de design}

Durante o planejamento do FleetTrack, foram consideradas diversas alternativas de design para garantir uma solução escalável, segura e de fácil manutenção. A adoção de uma arquitetura baseada em três camadas (frontend, backend e banco de dados) foi preferida devido à sua simplicidade e eficiência em projetos. 

Inicialmente, cogitou-se o uso de uma abordagem monolítica, pela facilidade de desenvolvimento e implantação, especialmente em projetos acadêmicos. Contudo, optou-se pela modularização através de microsserviços, visando maior escalabilidade e isolamento de falhas, facilitando a manutenção e futuras expansões.

O padrão MVC (Model-View-Controller) foi escolhido para o backend devido à sua ampla adoção, clareza na separação de responsabilidades e compatibilidade nativa com o framework Django.

\subsubsection*{3.2.2 Visão Inicial da Arquitetura}

A arquitetura do FleetTrack será composta pelos seguintes componentes principais e suas respectivas interconexões:

\begin{itemize}
    \item \textbf{Frontend:} Interface web desenvolvida com Django Templates, HTML, CSS e JavaScript, responsável pela interação com os usuários. Comunica-se com o backend via chamadas HTTP.
    \item \textbf{Backend:} API e lógica de negócios desenvolvida em Django, responsável pelo processamento de dados, regras de negócio e geração de relatórios.
    \item \textbf{Banco de Dados:} MySQL, destinado ao armazenamento persistente de informações relativas aos veículos, motoristas, trajetos e relatórios.
    \item \textbf{Serviço de Autenticação:} Implementação de autenticação e autorização baseada em Django Authentication, garantindo segurança na troca de informações.
\end{itemize}

A comunicação entre frontend e backend será realizada via APIs RESTful, utilizando o padrão HTTP, enquanto o backend se conecta ao banco de dados através do ORM (Object-Relational Mapping) do Django.

\subsubsection*{3.2.3 Padrões de Arquitetura}

O projeto adota os seguintes padrões arquiteturais:

\begin{itemize}
    \item \textbf{MVC (Model-View-Controller):} Para organizar a estrutura do backend, separando claramente as camadas de modelo de dados, lógica de negócio e apresentação.
    \item \textbf{RESTful APIs:} Para padronizar a comunicação entre frontend e backend, facilitando a integração com outros sistemas.
\end{itemize}
\newpage
\subsubsection*{3.2.4 Modelos C4}

Para uma melhor compreensão da arquitetura do FleetTrack, foi adotado o Modelo C4, que descreve o sistema em três níveis principais de abstração: Contexto, Contêineres e Componentes. Esses níveis possibilitam uma representação gradual da arquitetura, partindo de uma visão geral do sistema até o detalhamento das responsabilidades internas dos seus principais módulos

\paragraph{Contexto:}
Diagrama de Contexto(Figura 2), cuja finalidade é fornecer uma visão geral das interações do sistema com agentes externos. O FleetTrack é um sistema de gestão de frotas acessado por um único ator, o Administrador, que exerce funções operacionais e gerenciais através de uma interface web.

Além da interação humana, o sistema também realiza comunicação automatizada com o serviço externo OpenRouteService API, o qual fornece dados técnicos relacionados a rotas, como informações de distância, elevação e caminhos disponíveis. Essa integração tem como objetivo enriquecer funcionalidades internas, como o planejamento de itinerários e a geração de relatórios baseados em dados geográficos atualizados.

\begin{figure}[!ht]
    \centering
    \includegraphics[width=0.7\linewidth,height=1\textheight,keepaspectratio]
    {C4Contexto.png}
    \caption{C4 Contexto}
\end{figure}
\newpage
 
\paragraph{Contêineres:}
O sistema é dividido em três contêineres principais:

\begin{itemize}
    \item \textbf{Frontend Web:} Aplicação baseada em Django Templates com HTML/CSS/JS. Responsável por renderizar páginas e interagir com o backend via requisições HTTP.
    \item \textbf{Backend API:} Serviço em Django, que expõe endpoints RESTful, processa as regras de negócio, autentica usuários via Django Authentication e interage com o banco de dados.
    \item \textbf{Banco de Dados Relacional:} MySQL, responsável pelo armazenamento persistente e seguro dos dados da frota.
\end{itemize}

Diagrama de Contêineres(Figura 3), evidenciando os principais blocos de software que o compõem e suas respectivas interações. O sistema é dividido em três contêineres principais: o Frontend Web, a Backend API e o Banco de Dados.

O Frontend Web, desenvolvido com Django Templates, representa a camada de apresentação acessada pelo usuário por meio de um navegador. Esse frontend se comunica com a Backend API, desenvolvida com Django e arquitetura RESTful, responsável por processar as regras de negócio, realizar autenticação de usuários e integrar serviços externos.

A persistência dos dados é realizada em um banco de dados relacional MySQL, que armazena informações como usuários, veículos, rotas e configurações do sistema. Complementarmente, a Backend API realiza requisições ao serviço externo OpenRouteService API para obtenção de dados técnicos sobre rotas, promovendo maior precisão e eficiência nos processos logísticos oferecidos pela aplicação.


\begin{figure}[!ht]
    \centering
    \includegraphics[width=9\linewidth,height=0.8\textheight,keepaspectratio]
    {C4Container.png}
    \caption{C4 Container}
\end{figure}
 \newpage
 \begin{figure}[!hb]
    \centering
    \includegraphics[width=1.1\linewidth,height=1\textheight,keepaspectratio]
    {C4Componentes.png}
    \caption{C4 Componentes}
\end{figure}

\paragraph{Componentes:}
O Diagrama de Componentes, representado na Figura 4, descreve a estrutura interna dos contêineres, detalhando como eles são compostos por módulos funcionais. No caso do FleetTrack, esse diagrama foca no backend da aplicação, desenvolvido em Django, e identifica seus principais componentes lógicos.

Entre os componentes destacados estão o Serviço de Autenticação, o Módulo de Cadastro, que gerencia os dados de veículos; o Sistema de Geração de Relatórios, voltado à produção de informações gerenciais; e o Cliente de Integração Externa, que consome dados de rotas por meio de uma API pública. Esses componentes interagem entre si e com o banco de dados relacional, garantindo o funcionamento coeso do sistema.

\newpage
\subsection{Stack Tecnológica}

\subsubsection*{Linguagens de Programação:}
\begin{itemize}
\item \textbf{Frontend:} Django Templates (HTML, CSS, JavaScript) para renderização dinâmica no lado do servidor.
\item \textbf{Backend:} Django (Python) para desenvolvimento da aplicação.
\item \textbf{Banco de Dados:} MySQL para armazenamento estruturado de informações.
\end{itemize}

\subsubsection*{Ferramentas de Desenvolvimento e Gestão de Projeto:}
\begin{itemize}
\item \textbf{Git/GitHub}: para controle de versão e colaboração.
\item \textbf{Postman}: para testes de APIs.
\item \textbf{Google Cloud Platform (GCP):} Ambiente utilizado para o deploy e hospedagem da aplicação.
\end{itemize}


\subsection{Considerações de Segurança}

A segurança será um ponto central no desenvolvimento do sistema. Algumas medidas previstas:
\begin{itemize}
\item \textbf{Autenticação e autorização:} Controle de acesso utilizando Django Authentication.
\item \textbf{Proteção contra vulnerabilidades:} Medidas para evitar ataques como SQL Injection, Cross-Site Scripting (XSS) e Cross-Site Request Forgery (CSRF).
\end{itemize}

\newpage

\section{Próximos Passos}

Para garantir um desenvolvimento organizado e validado em todas as fases do projeto, os próximos passos estão estruturados conforme descrito a seguir:

\subsection{Criação de mockups e fluxos de navegação}
\begin{itemize}
    \item Elaboração dos mockups das telas principais do sistema.
    \item Definição dos fluxos de navegação entre as páginas.
\end{itemize}

\subsection{Desenvolvimento de protótipo}
\begin{itemize}
    \item Implementação de uma versão inicial com funcionalidades básicas.
    \item Testes preliminares com foco na navegação e fluxo de dados.
    \item Correção de eventuais falhas de usabilidade ou lógica.
\end{itemize}

\subsection{Implementação dos módulos}
\begin{itemize}
    \item Desenvolvimento dos principais componentes do sistema:
    \begin{itemize}
        \item Cadastro de motoristas e veículos;
        \item Gerenciamento de viagens;
    \end{itemize}
    \item Aplicação de boas práticas de segurança e desempenho.
\end{itemize}

\subsection{Testes e validações}
\begin{itemize}
    \item Realização de testes unitários.
    \item Validação de funcionalidades conforme requisitos do projeto.
    \item Ajustes com base em feedback dos testes.
\end{itemize}

\newpage

\section{Referências}
\begin{itemize}
\item \textbf{Django.} The web framework for perfectionists with deadlines. Disponível em: \url{https://www.djangoproject.com}. Acesso em: 14 fev. 2025.
\item \textbf{MySQL.} MySQL: The world's most advanced open source relational database. Disponível em: \url{https://www.mysql.com}. Acesso em: 8 fev. 2025.
\item \textbf{GitHub.} GitHub - Git repository hosting service. Disponível em: \url{https://github.com}. Acesso em: 29 fev. 2025.
\item \textbf{Postman.} Postman - API platform for building and using APIs. Disponível em: \url{https://www.postman.com}. Acesso em: 20 fev. 2025.
\end{itemize}

\end{document}
