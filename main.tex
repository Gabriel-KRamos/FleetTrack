\documentclass[12pt]{article}
\usepackage[utf8]{inputenc}
\usepackage[brazil]{babel}
\usepackage{graphicx}
\usepackage{url}

\begin{document}

\begin{titlepage}
    \centering

    {\Huge \textbf{Centro Universitário Católica de Santa Catarina}}\\[1.5cm]
    {\Large Curso de Engenharia de Software}\\[4cm]
    {\huge \textbf{FLEETTRACK}}\\[2cm]
    {\Large \textbf{Gabriel Knopka Ramos}}\\[6cm]
    {\Large Joinville, SC}\\[1.5cm]
    {\Large 2025}

    \vfill


\end{titlepage}

\newpage

\begin{abstract}
    \noindent \textbf{Resumo:} Este trabalho apresenta o desenvolvimento de um Sistema de Controle de Frotas com Rastreamento GPS, denominado FleetTrack, voltado para a otimização da gestão de veículos em empresas que necessitam monitoramento em tempo real. O projeto tem como objetivo criar uma solução eficiente e acessível para empresas que buscam melhorar a administração de suas frotas, reduzindo custos operacionais, aprimorando a logística e aumentando a segurança dos veículos e motoristas.
\end{abstract}

\vspace{1cm}

\section*{1 Introdução}
A gestão eficiente de frotas tem se tornado um desafio crescente para empresas de transporte, logística e setores que dependem do deslocamento contínuo de veículos. O avanço da tecnologia permitiu o desenvolvimento de sistemas capazes de monitorar frotas em tempo real, utilizando rastreamento GPS para otimizar rotas, reduzir custos operacionais e aumentar a segurança dos motoristas e das cargas transportadas. Nesse contexto, este projeto propõe o desenvolvimento de um sistema de controle de frotas com rastreamento GPS, oferecendo uma solução tecnológica para empresas que necessitam de maior controle e eficiência na administração de seus veículos.

O objetivo principal deste projeto é desenvolver um sistema de controle de frotas com rastreamento GPS que permita o monitoramento em tempo real dos veículos, proporcionando maior eficiência na gestão e reduzindo custos operacionais. Além disso, objetiva-se oferecer funcionalidades complementares, como geração de relatórios, alertas de eventos críticos e análise de desempenho da frota. Dessa forma, espera-se que o sistema contribua para a melhoria da logística e do planejamento estratégico das empresas que adotarem essa solução.

\newpage

\section*{2 Descrição do Projeto}

O presente projeto tem como tema o desenvolvimento de um sistema de controle de frotas com rastreamento GPS, focado na otimização da gestão de veículos para empresas que necessitam monitoramento contínuo e eficiente. O sistema proposto permitirá o acompanhamento em tempo real da localização dos veículos, proporcionando maior controle sobre suas operações e facilitando a tomada de decisões estratégicas. A ferramenta será desenvolvida com funcionalidades voltadas para o monitoramento da frota, geração de relatórios e envio de alertas para eventos críticos, visando aprimorar a logística e a segurança dos transportes.

O projeto busca resolver diversos problemas enfrentados por empresas que gerenciam frotas de veículos. Entre os principais desafios, destacam-se a dificuldade de monitoramento em tempo real, a falta de dados precisos sobre o desempenho da frota, o alto custo operacional devido à ineficiência de rotas e ao consumo excessivo de combustível, além da ausência de um controle centralizado para a gestão dos veículos. Com a implementação do sistema, pretende-se oferecer uma solução tecnológica que permita acompanhar a localização dos veículos, reduzir desperdícios, otimizar trajetos e aumentar a segurança tanto para motoristas quanto para a carga transportada.

No entanto, algumas limitações foram estabelecidas para este projeto. O sistema não abrangerá funcionalidades de manutenção preditiva dos veículos, não incluirá integração com sensores físicos embarcados para diagnóstico mecânico e não terá um módulo específico para análise de desempenho de motoristas baseado em comportamento ao volante. Além disso, a solução será desenvolvida para frotas terrestres, não sendo aplicável para meios de transporte aéreo ou marítimo. Dessa forma, o escopo do projeto se concentra na gestão de frotas rodoviárias, priorizando a eficiência operacional e o monitoramento por GPS.

\newpage


\section*{3 Especificação Técnica}

Este projeto tem como objetivo o desenvolvimento de um sistema de controle de frotas com rastreamento via GPS, destinado a empresas que necessitam de monitoramento contínuo e centralizado de seus veículos. O sistema permitirá o rastreamento em tempo real, proporcionando uma visão detalhada da localização dos veículos, a fim de otimizar a gestão da frota e melhorar a tomada de decisões. Além disso, contará com funcionalidades avançadas de geração de relatórios, que oferecerão informações valiosas sobre o desempenho da frota, consumo de combustível, tempo de inatividade e manutenção dos veículos. A otimização de rotas será um dos recursos-chave do sistema, visando aumentar a eficiência das operações, reduzir custos e melhorar a produtividade. O sistema será projetado para ser escalável, podendo atender tanto pequenas frotas quanto grandes empresas com centenas de veículos.

\subsection*{3.1 Requisitos de Software}

\subsubsection*{Requisitos Funcionais (RF)}
\begin{itemize}
    \item RF01: Permitir o cadastro e gerenciamento de veículos da frota.
    \item RF02: Rastrear a localização dos veículos em tempo real por meio de GPS.
    \item RF03: Exibir a posição dos veículos em um mapa.
    \item RF04: Permitir a criação e o gerenciamento de usuários não administradores.
    \item RF05: Fornecer um painel administrativo para gerenciar frotas, motoristas e rotas.
\end{itemize}

\subsubsection*{Requisitos Não Funcionais (RNF)}
\begin{itemize}
    \item RNF01: O sistema deve ser acessível via web.
    \item RNF02: A comunicação entre o GPS e o servidor deve ocorrer com baixa latência.
    \item RNF03: O sistema deve armazenar dados de rastreamento por pelo menos seis meses.
    \item RNF04: A interface deve ser intuitiva e de fácil uso para diferentes perfis de usuários.
    \item RNF05: O sistema deve notificar eventos críticos, como desvios de rota ou paradas não autorizadas.
    \item RNF06: O sistema deve gerar relatórios sobre trajetos percorridos, consumo de combustível e tempo de deslocamento.
\end{itemize}

\subsection*{3.2 Considerações de Design}

O design do sistema será planejado para garantir escalabilidade e facilidade de manutenção. Durante o desenvolvimento, serão analisadas diferentes alternativas arquiteturais, com a abordagem baseada em microsserviços sendo considerada para possibilitar modularidade e melhor gerenciamento de cargas.

O sistema será composto por três camadas principais:
\begin{itemize}
    \item \textbf{Frontend:} Responsável pela interface do usuário, permitindo acesso ao sistema via navegador.
    \item \textbf{Backend:} Processará requisições, armazenará dados e gerenciará a lógica de negócio.
    \item \textbf{Banco de Dados:} Guardará informações sobre os veículos, trajetos, eventos críticos e usuários.
\end{itemize}

\subsubsection*{Padrões de Arquitetura}
O sistema será estruturado utilizando o padrão MVC (Model-View-Controller) no backend para organizar a lógica do software e facilitar a manutenção. Além disso, o uso de APIs RESTful permitirá a comunicação eficiente entre o frontend e o backend.

\subsubsection*{Modelos C4}
A arquitetura do sistema será detalhada por meio do modelo C4, abordando os seguintes níveis:
\begin{itemize}
    \item \textbf{Contexto:} Definição de como o sistema interage com usuários e serviços externos.
    \item \textbf{Contêineres:} Descrição dos serviços e módulos principais do sistema.
    \item \textbf{Componentes:} Detalhamento dos elementos internos dos contêineres e suas funções.
    \item \textbf{Código:} Representação detalhada da implementação dos módulos principais.
\end{itemize}

\subsection*{3.3 Stack Tecnológica}

\subsubsection*{Linguagens de Programação:}
\begin{itemize}
    \item \textbf{Frontend:} React.js para desenvolvimento do frontend.
    \item \textbf{Backend:} Go para manipulação eficiente de dados e processamento de requisições.
    \item \textbf{Banco de Dados:} MySQL para armazenamento relacional estruturado.
\end{itemize}

\subsubsection*{Ferramentas de Desenvolvimento e Gestão de Projeto:}
\begin{itemize}
    \item Git/GitHub para versionamento e colaboração no código.
    \item Postman para testes de API.
\end{itemize}

\subsection*{3.4 Considerações de Segurança}

A segurança do sistema será tratada com prioridade para evitar vulnerabilidades e proteger os dados sensíveis. Algumas das práticas adotadas incluem:
\begin{itemize}
    \item \textbf{Autenticação e autorização:} Implementação de autenticação baseada em JWT (JSON Web Token) para controle de acessos.
    \item \textbf{Prevenção contra ataques:} Proteção contra SQL Injection, Cross-Site Scripting (XSS) e Cross-Site Request Forgery (CSRF).
\end{itemize}

\newpage

\section*{4 PRÓXIMOS PASSOS}

\newpage

\section*{Referências}

\begin{itemize}
    \item \textbf{React.js.} React - A JavaScript library for building user interfaces. Disponível em: \url{https://reactjs.org/}. Acesso em: 12 fev. 2025.
    \item \textbf{Leaflet.js.} Leaflet - JavaScript library for interactive maps. Disponível em: \url{https://leafletjs.com/}. Acesso em: 25 mar. 2025.
    \item \textbf{Mysql.} Mysql: The world's most advanced open source relational database. Disponível em: \url{https://www.mysql.com}. Acesso em: 8 abr. 2025.
    \item \textbf{JWT.io.} JWT (JSON Web Token) Introduction. Disponível em: \url{https://jwt.io/}. Acesso em: 19 mai. 2025.
    \item \textbf{Docker.} Docker - Empowering developers to write, test and deploy code anywhere. Disponível em: \url{https://www.docker.com/}. Acesso em: 3 jun. 2025.
    \item \textbf{GitHub.} GitHub - Git repository hosting service. Disponível em: \url{https://github.com/}. Acesso em: 29 jul. 2025.
    \item \textbf{Go Documentation.} The Go Programming Language Documentation. Disponível em: \url{https://golang.org/doc/}. Acesso em: 14 set. 2025.
    \item \textbf{Postman.} Postman - API platform for building and using APIs. Disponível em: \url{https://www.postman.com/}. Acesso em: 20 nov. 2025.
\end{itemize}


\end{document}